\documentclass[12pt]{article}
\usepackage{amsmath}
\usepackage{indentfirst}
\setlength{\parindent}{0em}
\usepackage{graphicx}
\usepackage{setspace}


%上角标
\newcommand{\rr}[1]{^{#1}}
%粗体字
\newcommand{\bb}[1]{{\textbf {#1}}}
%下划线
\newcommand{\uu}[1]{\underline{#1}}
%求极限
\newcommand{\infinity}{\infty}
%分式求导
\newcommand{\fp}[2]{\frac{\partial{#1}}{\partial{#2}}}
%加总公式
\newcommand{\addup}[4]{\sum\limits_{#1 = #2} ^#3 {#4}}
%连乘公式
\newcommand{\pai}[4]{\prod_{#1 = #2} ^#3 {#4}}
%求极限
\newcommand{\Lim}[3]{\lim_{#1 \to #2} {#3}}
%可调整大小的括号
\newcommand{\bkt}[1]{\left( {#1} \right)}
%积分
\newcommand{\integral}[3]{\int_{#1}^{#2}{#3}}
%行内公式
\newcommand{\e}[1]{$ #1 $}
%行间公式
\newcommand{\ee}[1]{$$ #1 $$}



\title{Customized Math Command}
\author{Synferlo}
\date{August 28, 2020}


\begin{document}
    \maketitle
    \newpage

    \section{Math Command}
    $$$$

        \subsection{Summation}
            % Command: \addup{arg1}{arg2}{arg3}{arg4}
            % arg1: 下标变量; arg2: 下标起始值;  arg3:加总上限(上标); arg4:加总号后面的公式
            $$ \addup{s}{0}{\infinity}{X_s} $$


        \subsection{Fraction-Partial-Derivative}
            % Command: \fp{arg1}{arg2}
            % arg1: 分子; arg2:分母
            $$ \fp{X_i}{P_i} $$

        
        \subsection{Production}
            % Command: \pai{arg1}{arg2}{arg3}{arg4}
            % arg1: 下标变量; arg2: 下标起始值;  arg3:连乘上限(上标); arg4:连乘号后面的公式
            $$ \pai{i}{1}{\infinity}{X_i} $$


        \subsection{Limitation}
            % Command: \Lim{arg1}{arg2}{arg3}
            % arg1: 变量; arg2:趋近值;   arg3:后续公式
            $$ \Lim{n}{\infinity}{\frac{\sigma^2}{n}} $$


        \subsection{Parenthese}
            $$ \bkt{\cfrac{\addup{i}{1}{\infinity}{X_i}}{\pai{i}{1}{\infinity}{Y_i}}}$$


        \subsection{Integral}
            % Command: \integral{arg1}{arg2}{arg3}
            % arg1: 下限; arg2:上限; arg3:积分内公式
            $$ f(x) = \integral{-\infinity}{\infinity}{F(x)}dx $$

        \subsection{Math Equation}
            % Command: \e{arg}
            % arg1:需要书写的公式
            This is for inline equation \e{f(x)}\\

            This is an example for newline equation \ee{
                f(x) = g(x)
                }

        \subsection{BoldFace}
            If you need to set a \bb{boldface}





\end{document}