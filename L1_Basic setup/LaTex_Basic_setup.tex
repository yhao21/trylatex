%这里还可以使用book, report, letter等class
%注意,在letter类里面,没有maketitle命令
%我们在latex中一般不会删除内容,而是注释掉,方便我们在后续调整文档类时重新启用
\documentclass{article}
%这里使用amsmath数学包,这样可以让自己使用begin{equation}和begin{aligned}格式化处理数学公式
\usepackage{amsmath}
%使用这个ctex包可以让文档识别中文,如果不使用,则需要将documentclass设置为ctexart
\usepackage{ctex}


%自定义自己的命令
%在导言区,也就是这里,定义自己的命令。比如,我想吧角度符号那个圆圈定义为 \degree:
%使用 \newcommand 命令进行自定义,后面接上你想定义的命令,花括号内填写这个命令对应的代码
\newcommand \degree{^\circ}


\title{Basic Setup for LaTex}
\author{SynFerLo}
\date{August 27, 2020}

%这里是正文区域
%s文档内容需要编写在begin和end document之间
%一个latex文件有且只能有一个document环境
\begin{document}

    %这里使用maketitle是为了显示我们在8-10行写的标题、作者、日期,若不添加maketitle
    %则第一页显示的就是begin{document}后的正文部分
    \maketitle

    \newpage

    %快捷输入数学公式的方法:
    %1. 行内公式: $ formula $      编辑后的公式与文本在同一行
    %2. 行间公式: $$ formula $$    编辑后的公式会在新的一行,并且自动居中
    
    This is an example for writing a formula in the same line.
    $U(X)=X_{1}^\alpha X_{2}^\beta$

    This is an example for writing a formula in a new line.
    $$U(X)=X_{1}^\alpha X_{2}^\beta$$

    %换行:
    %我们可以通过在代码区空一行进行pdf中的换行,注意换行后是自动缩进的

    %也可以通过 \\ 进行换行
    %注意,\\换行是不会进行首行缩进的!!
    Example 1:

    This is an example of switch line.

    This is the new line.\newline

    Example 2:

    This is an example of switch line.

    This is the new line.\newline

    \textbf{Example 3:}

    This is an example for self-designed command:
    %注意,书写任何数学公式都需要添加$, 不然可能会出现排版错误
    $\angle90\degree$


    If you need to type backslash in your file, use \textbackslash











\end{document}




