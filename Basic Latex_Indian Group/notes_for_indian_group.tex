\documentclass[12pt]{article}
\usepackage{amsmath}

%使用这个包和下面的命令可以设置整个文档的缩进
\usepackage{indentfirst}
\setlength{\parindent}{1.5em}




\title{Notes for Indian Group Instruction}
\author{Synferlo}
\date{August 29, 2020}


\begin{document}
    \maketitle

    \newpage
    \section{Layout Formatting}
        \subsection{Indentation}
            %换行的话只需要留出一个空行即可,但是注意,这种方法换行后会默认首行缩进!
            %如不想 “首行缩进”, 在每一段段首 使用 \noindent 命令
            
            Example with indentation:

            You might have noticed that even though the length of the lines of text we type in a
            paragraph are different, in the output, all lines are of equal length, aligned perfectly on
            the right and left. TEX does this by adjusting the space between the words.\\

            Example without indentation:

            \noindent You might have noticed that even though the length of the lines of text we type in a
            paragraph are different, in the output, all lines are of equal length, aligned perfectly on
            the right and left. TEX does this by adjusting the space between the words.

        \subsection{Quotes}
            % quotes 引号

        \subsection{Dash lines}
            % 通过输入不同数量的 - 控制长短
            X-ray are discussed in pages 221--255 of Volume 3---the volume on electromagnetic waves.
        
        \subsection{Backlash}
            % 使用 \textbackslash 输出斜线。
            Use \textbackslash textbackslash to produce \textbackslash.

        \subsection{Special symbols}
            % 使用 \ 加符号的形式 输出特定符号
            \noindent 90 \% of the people\\
            \# \\
            \$30 \\
            \_ \\
            \& \\
            \{  \} \\
            \textasciitilde\\
            \textasciicircum

        
        \subsection{Line-Distance}
            % 可使用 \\[10pt] 来设定单独一行的行间距
            This is the first line.\\[15pt]
            This is the second line.\\
            This is the thrid line without setting distance.

        
        \subsection{Text Positioning}
            % 使用 center环境 完成居中, 使用[.75cm]规定行间距, 使用flushright规定右对齐, flushleft左对齐
            \begin{center}
                Clemson University\\[.75cm]
                Certificate
            \end{center}
            
            \noindent This is to certify that Synferlo has finish this program.

            \begin{flushright}
                The Director\\
                SynFerlo
            \end{flushright}

    \section{Fonts}

        \subsection{Type style}
            See font notes.
        
        \subsection{Emphasis and Bold}
            % 使用 \emph 完成强调命令,他会将文本变成与周围不同的字体。
            % 比如正文是正常的upright字体,\emph 会把单词变成斜体
            % 如果正文是斜体, \emph 会把单词变成upright
            \textit{This is a test about \emph{emphasis}}.\\
            \indent This is a test about \emph{emphasis}.
        
            {\bfseries This is a good learning} opportunity.
    
    \section{document}
    
        \subsection{Document Class}
            % \documentclass[]{}
            % [] 写参数,如字号,纸张大小 a4paper, a5paper, b5paper
            Setup documentclass using \textbackslash documentclass [ ] \{ \}

            
         
            




\end{document}