\documentclass{article}
\usepackage{ctex}

%设置斜体+变粗+罗马族
\newcommand{\myfont}{\textbf{\textit{hello}}}


\title{Notes for font-change in LaTex}
\author{Synferlo}
\date{August 28, 2020}

\begin{document}

\maketitle
    \newpage
    LaTex中一个字体有5种属性:\\

    1. 字体编码\\


    2. 字体族(罗马字体、无衬线字体、打字机字体)

    %begin document下方设置字体后,则全局通用,也可以通过在每个段落前添加字体命令来修改当前段落的字体
    字体设置命令:\\

    方式一:

    \textrm{Roman Family} 罗马字体
    
    \textsf{Sans Serif Family} 无衬线字体


    \texttt{typewriter Family} 打字机字体\\

    方式二:

    \rmfamily Roman Family 罗马字体

    \sffamily Sans Serif Family 无衬线字体

    \ttfamily Typewriter Family 打字机字体\\

    通过每一段前添加命令修改段落字体:

    \rmfamily This is an Example for Roman Family paragraph.

    \ttfamily This is an Example for Typewriter Family paragraph.\\


    3. 字体系列(粗细,宽度)\\

    \textmd{Medium Series} 一般样式
    
    \textbf{Boldface Series} 粗体样式

    {\mdseries Medium Series} 一般样式

    {\bfseries Boldface Series} 粗体样式\\%全局粗体

    4. 字体形状(直立,斜体,伪斜体,小型大写)\\

    \textup{Upright Shape} 直立字体

    \textit{Italic Shape} 斜体

    \textsl{Slanted Shape} 伪斜体

    \textsc{Small Caps Shape} 小型大写\\


    5. 中文字体设置:\\

    需要使用ctex包后才能设置中文字体!
    {\songti 宋体}

    {\heiti 黑体}

    {\fangsong 仿宋}

    {\kaishu 楷书}

    6. 字体大小\\

    整体文档的字体大小需要在documentclass[字号]{article}中进行设置。

    比如:documentclass[10pt]{article} 表示全局使用10pt的大小作为标准字体

    一般只有10,11,12三种字号

    设置完全局后,若要在某处单独修改字号,可通过以下方式:
    
    {\tiny          Hello}\\
    {\scriptsize    Hello}\\
    {\footnotesize  Hello}\\
    {\small         Hello}\\

    This is Normal Hello\\

    {\normalsize    Hello}\\
    {\large         Hello}\\
    {\Large         Hello}\\
    {\LARGE         Hello}\\
    {\huge          Hello}\\
    {\Huge          Hello}\\

    对于中文命令,ctex专门设置了特殊的命令:\\

    {\zihao{5}   你好}

    {\zihao{-0}  你好}\\

    注意:一般而言我们不会在正文中频繁使用切换字体、字号的命令,而是会以newcommand的形式在导言区
    预先设置好字体、字号、样式的组合,而后直接在正文中使用新的命令引用这个组合。\\

    现在来引用我们设置好的字体命令:myfont

    \myfont{This is the font designed for myself.}








\end{document}